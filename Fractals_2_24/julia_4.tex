
\documentclass[12pt]{article}
\usepackage{amssymb,amscd}
\usepackage{euscript}

\textwidth19cm
\hoffset=-2cm
\addtolength{\textheight}{2cm}
%\leftmargin 3cm
%\rightmargin 3cm
\voffset=-3cm

\def\R{\mathbb R}
\def\C{\mathbb C}

\begin{document}
\setlength{\parindent}{0pt}


\hskip4cm 
{\Large { \bf {Julia Sets}} }

\vskip.7cm

{\large
The {\bf Julia set} $J_c=J(f_c)\,$ is the boundary of the set of points  
\vskip.05cm
whose  orbits under the map $\,f_c(z)=z^2+c\,$ are bounded.  


\vskip.5cm

{\bf Properties of Julia sets.}\,   For every $c\in \C$
\vskip.1cm
$\circ$  \, $J_c$ is non-empty, 
\vskip.1cm
$\circ$  \, $J_c$ is compact, 
% $\circ$  \, $J_c$ is non-empty, closed and bounded, 
\vskip.1cm
$\circ$ \, $J_c$ contains no isolated points,
\vskip.1cm
$\circ$ \, $f_c(J_c)=J_c\,\,$ and $\,\,f_c^{-1}(J_c)=J_c$.

\vskip.8cm

{\bf Relation between the Mandelbrot set and Julia sets:}
\vskip.2cm
\hskip1cm $c\in \bf M \;\Longleftrightarrow \;$  the set  $J_c$ is connected.
\vskip.5cm

Moreover,\, if $c\notin \bf M$,\, then $J_c$  is totally disconnected.
}
\vskip.5cm 
{\bf Note.} A set $X$ is {\it totally disconnected} if it  has no connected subsets 

consisting  of more than one point. 

Examples of closed totally disconnected sets without isolated points are 

the Cantor set  in $\R$\, and \,``Cantor dust" in $\R^2$.
\end{document}











