\documentclass[12pt]{amsart}
\usepackage{amssymb,amscd}
\usepackage{euscript}
%\userpackage{mathtools}


\textheight25.5cm
\textwidth17.9cm
\hoffset=-2.7cm
\voffset=-2.8cm
\addtolength{\textheight}{1cm}
\leftmargin 3cm
\rightmargin 3cm

\def\u{\bold{u}}
\def\R{\mathbb R}
\def\Q{\mathbb Q}
\def\N{\mathbb N}
\def\Z{\mathbb Z}
\def\T{\mathbb T}
\def\O{\Omega}
\def\o{\omega}
\def\e{\varepsilon}
\def\la{\lambda}
\def\a{\alpha}
\def\b{\beta}
\def\det{\text{det}}
\def\e{\epsilon}
\def\B{\EuScript{B}}
\def\A{\EuScript{A}}


\pagestyle{empty}

\def\QED{\hfill\hfill{\square}} 

\begin{document}

{\Large {\bf  Box(-counting) dimension.}}

\vskip.4cm

{\large  A box is an interval in $\R$, a square in $\R^2$, a cube in $\R^3$, ... .
\vskip.1cm 
 How many boxes with side $\e$ are needed to cover a bounded set $X$ in $\R^k$\,?

Denote this number $N(\e)$.
\vskip.2cm 
$X=$ unit interval \,  \hskip1cm   $X=$ unit square   \hskip1cm   $X=$ unit cube\,   \hskip1.5cm   In each case,\, 
\vskip.1cm 
$N(\e)\approx (1/\e)^1$   \hskip1.9cm  $N(\e)\approx (1/\e)^2$   \hskip1.4cm \,$N(\e)\approx (1/\e)^3$
\hskip1.8cm 
$d\approx$ {\Large $\frac{\ln N(\e)}{\ln (1/\e)}$}.
\vskip.3cm 

{\bf Def.} The {\bf box(-counting) dimension}\, of a bounded set $X$ in $\R^k$  is\, 
\vskip.1cm 
\hskip2.5cm $\dim _B(X)=\underset{\e\to 0^+}{\lim} \;${\Large $\frac{\ln N(\e)}{\ln (1/\e)}$},\, \, \,{\it if the limit exists, }
\vskip.1cm 
\hskip1.3cm where $N(\e)$ is the least number of boxed with side $\e$ needed to cover $X$.
\vskip.2cm 
\hskip.5cm  If the limit does not exist, we consider the upper and lower box dimensions,  
\vskip.1cm 
 \hskip.5cm  $\overline{\dim}_B(X)$ and 
$\underline{\dim}_B(X)$, defined as the corresponding $\liminf$ and $\limsup.$
\vskip.4cm 
$\bullet$\,  If the limit along a decreasing sequence $\e_n\to 0\,$ with $\e_{n+1}\ge c\e_n$\, equals $d$, 
\vskip.1cm 
\hskip.5cm  then  $\dim _B(X)=d.$
\vskip.4cm 

{\bf Example.}  $C =$ the Cantor set.
\vskip.1cm 
\hskip.5cm  At step $n$, the set $C_n$ consists of $2^n$ intervals of length $1/3^n$.\, 

 \vskip.1cm 
\hskip.5cm  So we take $\e_n=1/3^n$.
  Then $N(\e_n)=2^n$\, and \,\,{\Large $\frac{\ln N(\e_n)}{\ln (1/\e_n)}$} = {\Large $\frac{\ln 2^n}{\ln 3^n}$} =
 {\Large $\frac{\ln 2}{\ln 3}$}. \, 

\hskip.5cm
 Hence $\dim_B(C)=$ {\Large $\frac{\ln 2}{\ln 3}$} = $d_{sim}(C)$.

\vskip.4cm 
$\bullet$  For the examples we considered,\, it also holds that $\,\dim_B=d_{sim}$.
\vskip.05cm 
\hskip.6cm {\it Find their similarity and box dimensions.}

\vskip.4cm 
$\bullet$ Box dimension: pluses and minuses.
\vskip.15cm
\hskip.5cm $(+)$\, Relatively easy to define and to compute/estimate.
\vskip.15cm
\hskip.5cm $(-)$\, The limit does not necessarily exist, and we may only get liminf and limsup.
\vskip.15cm
\hskip.5cm $(-)$\, A countable set may have positive box dimension {\it (see next slide)},
\vskip.02cm
\hskip1.4cm and so adding countably many points to a set can change its box dimension.

\vskip.7cm 


$\bullet$\, {\bf Hausdorff dimension},\,  $\dim _H(X)$,\, does not have these drawbacks, 
\vskip.02cm
\hskip.5cm  but it is harder to define and compute/estimate.
\vskip.1cm
\hskip.5cm  The construction involves covering $X$ by balls of diameter {\it at most $\e$}.

\vskip.3cm 


$\bullet$\,  For any bounded set $X\subset \R^k$, \, $\dim _H(X) \le \underline{\dim}_B(x) \le \overline{\dim}_B(x)\le k$.
\vskip.3cm 

$\bullet$\, For the sets that we considered, \, $\dim _H(X) = \dim_B(X) = \dim_{sim}(X)$.

}


 \end{document}

 


%%%%%%%%%%%%%%%%%%%%%


     
            

